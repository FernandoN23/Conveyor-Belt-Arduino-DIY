\section{Introduction}

\par Thank you for choosing the REPOS Sorting Conveyor Belt!
Designed by Fernando Navarrete, this innovative solution transforms the way objects are sorted. With its advanced color detection system and precise integration of electromechanical actuators, you can automate sorting routines efficiently and reliably. \\

\par Figure \ref{fig:1-Correa_armada} shows the fully assembled system, ready to optimize your sorting processes. Its design allows handling different sizes and types of objects by adjusting the speed and sequence of the actuators according to your specific needs. Additionally, its modular construction facilitates maintenance and future expansions, ensuring consistent performance and adaptability to various environments. 

\begin{figure}[htbp]
    \centering
    \includegraphics[width=0.9\textwidth]{img/1-Ensamble Correa Transportadora Oruga}
    \caption{REPOS Sorting Conveyor Belt}
    \label{fig:1-Correa_armada}
\end{figure}

\par In the following sections, the steps required to assemble your REPOS Sorting Conveyor Belt will be detailed.


\clearpage
\section{Materials and Parts List}

\par First, the following materials are required to assemble the conveyor belt, as detailed in Table \ref{tab:2-materials}.

\begin{table}[H]
\centering
\caption{Materials list per conveyor belt}
\begin{tabular}{|l|l|c|}
    \hline
    \textbf{Name} & \textbf{Description} & \textbf{Quantity} \\
    \hline
    M3x8 Bolts & ISO M3 cylindrical head bolts & 40 \\
    \hline
    M4x8 Bolts & ISO M4 cylindrical head bolts & 8 \\
    \hline
    M3x4 Bolts & ISO M3 flat head bolts & 2 \\
    \hline
    627zz Bearings & SKF radial bearings 7x22x7mm & 4 \\
    \hline
    DC Motor & Yellow 6V DC motor with gearbox & 1 \\
    \hline
    L298N Driver & Motor driver L298N & 1 \\
    \hline
    GY-31/TCS3200 Sensor & 64-photodiode color sensor & 1 \\
    \hline
    SG90 Servo & SG90 servomotors with horns & 2 \\
    \hline
    18650 Battery & 3.7V lithium-ion 18650 battery & 2 \\
    \hline
    2x18650 Battery Holder & 18650 lithium-ion battery holder & 1 \\
    \hline
    SPST 2-pin Switch & 20mm diameter SPST switch & 1 \\
    \hline
    Neodymium Magnets & 12x4mm neodymium magnets for bearing covers & 4 \\
    \hline
\end{tabular}
\label{tab:2-materials}
\end{table}

\par Next, the list of 3D-printed parts required to assemble the conveyor belt is shown in Table \ref{tab:2-piezas}.

\begin{table}[H]
    \centering
    \caption{3D-printed parts list for each conveyor belt}
    \begin{tabular}{|l|l|c|}
        \hline
        \textbf{Name} & \textbf{Description} & \textbf{Quantity} \\
        \hline
        Track Chain & 43-link track chain & 1 \\
        \hline
        Front Motor Gear & Gear of the left power unit & 1 \\
        \hline
        Rear Motor Gear & Gear of the right power unit & 1 \\
        \hline
        Motor Mount & DC motor support piece & 1 \\
        \hline
        Cylinder & Support unit cylinder & 1 \\
        \hline
        7mm Shaft & 7mm shaft for the support unit & 1 \\
        \hline
        Acrylic Supports & Transparent acrylic side supports & 2 \\
        \hline
        Side Wall Supports & Support pieces for acrylic side supports & 4 \\
        \hline
        Lower Support & Component support pieces & 2 \\
        \hline
        Component Plate & Plate for L298N driver and batteries & 1 \\
        \hline
        Servo Mount & Mounts for SG90 servos & 2 \\
        \hline
        Servo Blades & Push blades for SG90 servos & 2 \\
        \hline
        Side Receiving Box & Boxes for sorted objects & 2 \\
        \hline
        Top Support & Piece connecting box and servos to keep belt level & 2 \\
        \hline
        Dark Chamber & Darkened chamber for color sensor & 1 \\
        \hline
        Slat Curtains & Curtains for dark chamber & 2 \\
        \hline
        Loading Hopper & Hopper to direct objects to dark chamber & 1 \\
        \hline
        Bearing Covers & Covers to protect bearings and prevent shafts from coming out & 4 \\
        \hline
    \end{tabular}
    \label{tab:2-piezas}
\end{table}


\clearpage
\section{Assembly Process}

\par First, place the 627zz bearings into the acrylic side supports in the 22mm holes, as shown in Figure \ref{fig:3-armado_1}. 

\begin{figure}[H]
    \centering
    \includegraphics[width=0.9\textwidth]{img/3-Armado_1}
    \caption{Rear acrylic support with embedded bearings. Glue is recommended for a press-fit.}
    \label{fig:3-Armado_1}
\end{figure}

\par The bearings should protrude toward the interior of the conveyor belt, according to Figure \ref{fig:3-armado_2}.

\begin{figure}[H]
    \centering
    \includegraphics[width=0.8\textwidth]{img/3-Armado_2}
    \caption{Acrylic supports with bearings positioned in the final conveyor belt assembly.}
    \label{fig:3-Armado_2}
\end{figure}

\par Next, to facilitate assembly, bolt the components from the rear side and then join with the other acrylic support. Attach the front and rear motor gears to the DC motor along with the motor mount, forming the \textbf{power unit} shown in Figure \ref{fig:3-Armado_3}.

\begin{figure}[H]
    \centering
    \includegraphics[width=0.9\textwidth]{img/3-Armado_3}
    \caption{Assembled conveyor belt power unit, no additional fasteners required.}
    \label{fig:3-Armado_3}
\end{figure}

\par Then, insert the power unit into the left bearing of the rear side support using the shaft protruding from the gear, as shown in Figure \ref{fig:3-Armado_4}.

\begin{figure}[H]
    \centering
    \includegraphics[width=0.9\textwidth]{img/3-Armado_4}
    \caption{Power unit installed.}
    \label{fig:3-Armado_4}
\end{figure}

\par To secure the power unit, attach the lower support to the two lower 3mm holes near the unit, as shown in Figure \ref{fig:3-Armado_5}. 

\begin{figure}[H]
    \centering
    \includegraphics[width=0.9\textwidth]{img/3-Armado_5}
    \caption{Power unit bolted with lower support.}
    \label{fig:3-Armado_5}
\end{figure}

\par Repeat this step for the lower support on the other side of the acrylic support, see Figure \ref{fig:3-Armado_6}.

\begin{figure}[H]
    \centering
    \includegraphics[width=1\textwidth]{img/3-Armado_6}
    \caption{Lower supports bolted to acrylic support.}
    \label{fig:3-Armado_6}
\end{figure}

\par The component plate should be bolted together with the L298N driver using M3x8mm bolts and the 2x18650 battery holder using M3x4mm flat head bolts to allow battery placement, see Figure \ref{fig:3-Armado_7}.

\begin{figure}[H]
    \centering
    \includegraphics[width=0.8\textwidth]{img/3-Armado_7}
    \caption{Assembled component plate.}
    \label{fig:3-Armado_7}
\end{figure}

\par Next, attach the component plate to the assembly by bolting it to both lower supports, as shown in Figure \ref{fig:3-Armado_8}.

\begin{figure}[H]
    \centering
    \includegraphics[width=1\textwidth]{img/3-Armado_8}
    \caption{Component plate incorporated into the conveyor belt assembly.}
    \label{fig:3-Armado_8}
\end{figure}

\par \textbf{IMPORTANT:} After this step, check the initial connections of the DC motor, L298N driver, and power supply. Leave an extra wire available for the future connection of the SPST switch. Immediately connect the three control pins of the DC motor (ENA, IN1, and IN2). Also, reserve a wire for the 5V and GND return to power the servomotors. \\

\par Next, incorporate the \textbf{support unit} of the conveyor belt, consisting of the 7mm shaft and cylinder. Press-fit the 7mm shaft into the bearing opposite the power unit and place the cylinder inside the shaft, as shown in Figure \ref{fig:3-Armado_9}.

\begin{figure}[H]
    \centering
    \includegraphics[width=0.8\textwidth]{img/3-Armado_9}
    \caption{Support unit installed into the conveyor belt.}
    \label{fig:3-Armado_9}
\end{figure}

\par To complete the rear side acrylic support, place the servomotors into the upper holes of the acrylic supports. First, attach the servos without horns to the supports using M3 bolts, see Figure \ref{fig:3-Armado_10}. Pass the cables through the cutout in the support before positioning.

\begin{figure}[H]
    \centering
    \includegraphics[width=0.6\textwidth]{img/3-Armado_10}
    \caption{Servo mount with SG90 servo bolted.}
    \label{fig:3-Armado_10}
\end{figure}

\par Then, attach both servomotors to the current assembly, as illustrated in Figure \ref{fig:3-Armado_11}.

\begin{figure}[H]
    \centering
    \includegraphics[width=1\textwidth]{img/3-Armado_11}
    \caption{Servomotors installed in the conveyor belt assembly.}
    \label{fig:3-Armado_11}
\end{figure}

\par Finally, add the side wall supports to both acrylic supports using M4x8mm bolts, see Figure \ref{fig:3-Armado_12}.

\begin{figure}[H]
    \centering
    \includegraphics[width=1\textwidth]{img/3-Armado_12}
    \caption{Side wall supports attached to the acrylic support.}
    \label{fig:3-Armado_12}
\end{figure}

\par Next, carefully place the track chain between the teeth of the power unit gears and over the cylinder of the support unit, according to Figure \ref{fig:3-Armado_13}.

\begin{figure}[H]
    \centering
    \includegraphics[width=1.2\textwidth]{img/3-Armado_13}
    \caption{Track chain installed in the assembly.}
    \label{fig:3-Armado_13}
\end{figure}

\par Then, place the remaining side acrylic support over the shafts of the power unit and support unit, and bolt the lower supports to the acrylic support, as shown in Figure \ref{fig:3-Armado_14}.

\begin{figure}[H]
    \centering
    \includegraphics[width=1\textwidth]{img/3-Armado_14}
    \caption{Remaining side acrylic support installed.}
    \label{fig:3-Armado_14}
\end{figure}

\par \textbf{IMPORTANT:} At this stage, connect the 2-pin SPST switch using the previously reserved wire, attaching it to the positive terminal of the batteries to control the conveyor belt power. Route the servomotor connections through the opposite hole of the switch support, ensuring all wires face the front of the conveyor for future connections. The switch installed in the 20mm hole is shown in Figure \ref{fig:3-Armado_15}.

\begin{figure}[H]
    \centering
    \includegraphics[width=0.8\textwidth]{img/3-Armado_15}
    \caption{2-pin SPST switch installed in the assembly.}
    \label{fig:3-Armado_15}
\end{figure}

\par Next, place the bearing covers over the 4 bearings. These covers include pre-attached 12x4mm neodymium magnets with glue, as shown in Figure \ref{fig:3-Armado_16}.

\begin{figure}[H]
    \centering
    \includegraphics[width=0.8\textwidth]{img/3-Armado_16}
    \caption{Bearing covers installed.}
    \label{fig:3-Armado_16}
\end{figure}

\par Then, attach the servo blades (remember to calibrate the zero position of the servos before installation to ensure correct operation range), see Figure \ref{fig:3-Armado_17}.

\begin{figure}[H]
    \centering
    \includegraphics[width=0.8\textwidth]{img/3-Armado_17}
    \caption{Servo blades installed.}
    \label{fig:3-Armado_17}
\end{figure}

\par Next, assemble the dark chamber with the color sensor and the slat curtains before installing it on the conveyor belt. First, position the slat curtains into the upper holes of the chamber, as shown in Figure \ref{fig:3-Armado_18}.

\begin{figure}[H]
    \centering
    \includegraphics[width=0.8\textwidth]{img/3-Armado_18}
    \caption{Slat curtains positioned in the dark chamber.}
    \label{fig:3-Armado_18}
\end{figure}

\par Then, bolt the GY-31/TCS3200 color sensor into the chamber using M3 bolts through the slat curtains, as shown in Figure \ref{fig:3-Armado_19}. Note the cutouts for the sensor connection pins at the top of the chamber.

\begin{figure}[H]
    \centering
    \includegraphics[width=0.8\textwidth]{img/3-Armado_19}
    \caption{Color sensor GY-31/TCS3200 installed in the dark chamber.}
    \label{fig:3-Armado_19}
\end{figure}

\par After securing the sensor, bolt the dark chamber to the conveyor belt on the side holes, as illustrated in Figure \ref{fig:3-Armado_20}. M3x10mm bolts with nuts are recommended for better adjustment, but M3x8mm bolts also work.

\begin{figure}[H]
    \centering
    \includegraphics[width=1\textwidth]{img/3-Armado_20}
    \caption{Dark chamber installed on the conveyor belt.}
    \label{fig:3-Armado_20}
\end{figure}

\par Next, attach the loading hopper to the protrusions on the sides of the dark chamber, as shown in Figure \ref{fig:3-Armado_21}.

\begin{figure}[H]
    \centering
    \includegraphics[width=0.8\textwidth]{img/3-Armado_21}
    \caption{Loading hopper attached to the dark chamber.}
    \label{fig:3-Armado_21}
\end{figure}

\par Finally, attach the side receiving boxes along with the top support. First, pre-fix the receiving boxes with the top support for stability, as shown in Figure \ref{fig:3-Armado_22}.

\begin{figure}[H]
    \centering
    \includegraphics[width=0.8\textwidth]{img/3-Armado_22}
    \caption{Side receiving box pre-fixed with top support.}
    \label{fig:3-Armado_22}
\end{figure}

\par Then, slightly lift the track chain and slide both receiving boxes onto the pre-fixed top support. Ensure they do not interfere with the servomotors. Attach M3 bolts to the three holes on the top support (2 front, 1 back), as shown in Figures \ref{fig:3-Armado_23} and \ref{fig:3-Armado_24}.

\begin{figure}[H]
    \centering
    \includegraphics[width=1\textwidth]{img/3-Armado_23}
    \caption{Receiving boxes and top support assembly, front view.}
    \label{fig:3-Armado_23}
\end{figure}

\begin{figure}[H]
    \centering
    \includegraphics[width=1\textwidth]{img/3-Armado_24}
    \caption{Receiving boxes and top support assembly, rear view.}
    \label{fig:3-Armado_24}
\end{figure}

\par Optionally, tensioners can be installed on the lower part of the conveyor belt next to each lower support. This ensures proper tension so the track chain engages with the power unit gears and moves correctly. To install the tensioners, lift the return section of the chain, place the tensioners, and bolt them at their single attachment point, as shown in Figure \ref{fig:3-Armado_25}.

\begin{figure}[H]
    \centering
    \includegraphics[width=0.8\textwidth]{img/3-Armado_25}
    \caption{Tensioners bolted to the conveyor belt. The track chain is hidden in the CAD model as its bending occurs only in reality.}
    \label{fig:3-Armado_25}
\end{figure}

\par Additionally, the final receiving box at the end of the conveyor is optional, as shown at the beginning of this manual in Figure \ref{fig:1-Correa_armada}.
