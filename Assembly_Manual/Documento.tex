\section{Introducción}

\par Gracias por adquirir la Correa Transportadora Clasificadora de REPOS.
Una creación más de Fernando Navarrete. Este novedoso sistema permitirá ejecutar una rutina
de clasificación de objetos mediante la detección de colores y uso de actuadores electromecánicos.
\\

\par El sistema armado de la correa transportadora clasificadora de REPOS se muestra en la Figura \ref{fig:1-Correa_armada}

\begin{figure}[htbp]
    \centering
    \includegraphics[width=1\textwidth]{img/1-Ensamble Correa Transportadora Oruga}
    \caption{Correa Transportadora Clasificadora de Repos}
    \label{fig:1-Correa_armada}
\end{figure}

\par En las siguientes secciones, se detallarán los pasos a seguir para armar su Correa Transporadora Clasificadora de Repos.


\clearpage
\section{Listado de materiales y de piezas}

\par Primero, se requieren los siguientes materiales para el armado de la correa transportadora, estos se detallan en la Tabla \ref{tab:2-materiales}.

\begin{table}[H]
\centering
\caption{Listado de materiales por correa}
\begin{tabular}{|l|l|c|}
    \hline
    \textbf{Nombre} & \textbf{Descripción} & \textbf{Cantidad} \\
    \hline
    Pernos M3x8 & Pernos cabeza cilíndrica norma ISO M3 & 40 \\
    \hline
    Pernos M4x8 & Pernos cabeza cilíndrica norma ISO M4 & 8 \\
    \hline
    Pernos M3x4 & Pernos cabeza plana norma ISO M3 & 2 \\
    \hline
    Rodamientos 627zz & Rodamientos radiales norma SKF 7x22x7mm & 4 \\
    \hline
    Motor DC & Motor DC con gearbox amarillo de 6V & 1 \\
    \hline
    Driver L298N & Controlador de motores L298N & 1 \\
    \hline
    Sensor GY-31/TCS3200 & Sensor a color de 64 fotodiodos & 1 \\
    \hline
    Servo SG90 & Servomotores modelo SG90 con horquillas & 2 \\
    \hline
    Batería 18650 & Batería Ion de Litio formato 18650 de 3.7V & 2 \\
    \hline
    Portapilas 2x18650 & Portapilas de Ion de Litio formato 18650 & 1 \\
    \hline
    Switch SPST 2 pines & Switch SPST de 2 pines y 20mm de diametro & 1 \\
    \hline
    Imanes de neodimio & Imanes de neodimio 12x4mm para tapas cubrerodamientos & 4 \\
    \hline
\end{tabular}
\label{tab:2-materiales}
\end{table}

\par Luego, el listado de piezas impresas necesario para ensamblar la correa transportadora, se muestra en la Tabla \ref{tab:2-piezas}.

\begin{table}[H]
    \centering
    \caption{Listado de piezas impresas en 3D para cada correa}
    \begin{tabular}{|l|l|c|}
        \hline
        \textbf{Nombre} & \textbf{Descripción} & \textbf{Cantidad} \\
        \hline
        Cadena de oruga & Cadena tipo oruga 43 eslabones & 1 \\
        \hline
        Engranaje Motor Frontal & Engranaje de la unidad de potencia izquierdo & 1 \\
        \hline
        Engranaje Motor Trasero & Engranaje de la unidad de potencia derecho & 1 \\
        \hline
        Soporte Motor & Pieza de soporte del motor dc & 1 \\
        \hline
        Cilindro & Cilindro de la unidad de soporte & 1 \\
        \hline
        Eje 7mm & Eje de 7mm de la unidad de soporte & 1 \\
        \hline
        Soportes de acrílico & Soportes laterales de la correa de acrílico transparente & 2 \\
        \hline
        Apoyos paredes laterales & Piezas de apoyo para los soportes laterales de acrílico & 4 \\
        \hline
        Soporte inferior & Piezas de soporte para componentes electrónicos & 2 \\
        \hline
        Placa de componentes & Placa para driver L298N y baterías & 1 \\
        \hline
        Soporte servomotor & Soportes para servos SG90 & 2 \\
        \hline
        Pala servomotores & Pala para empuje servos SG90 & 2 \\
        \hline
        Caja receptora lateral & Cajas receptoras de objetos clasificados & 2 \\
        \hline
        Soporte Superior & Pieza de unión caja y servos que mantiene correa nivelada & 2 \\
        \hline
        Cámara Oscura & Cámara oscurecida para sensor a color & 1 \\
        \hline
        Cortinas de Lamas & Cortinas de lamas para cámara oscura & 2 \\
        \hline
        Tolva de carga & Tolva para dirigir objetos a cámara oscura & 1 \\
        \hline
        Tapas cubrerodamientos & Tapas para cubrir rodamientos y evitar que los ejes salgan & 4 \\
        \hline

    \end{tabular}
    \label{tab:2-piezas}
\end{table}


\clearpage
\section{Proceso de armado}

\par Primero, se deben colocar los rodamientos 627zz en los soportes laterales de acrílico en las perforaciones de 22mm, tal como se muestra en la Figura \ref{fig:3-armado_1} . 

\begin{figure}[H]
    \centering
    \includegraphics[width=0.9\textwidth]{img/3-Armado_1}
    \caption{Soporte de acrílico trasero con rodamientos incrustados. Se recomienda usar pegamento para fijar a presión.}
    \label{fig:3-armado_1}
\end{figure}

\par Los rodamientos deben sobresalir hacia el interior de la correa transportadora, de acuerdo a la Figura \ref{fig:3-armado_2}.

\begin{figure}[H]
    \centering
    \includegraphics[width=0.8\textwidth]{img/3-Armado_2}
    \caption{Soportes de acrílico con rodamientos y dispuestos en la posición final de la correa transportadora.}
    \label{fig:3-armado_2}
\end{figure}

\par Luego, para facilitar el ensamblado, se empernarán los componentes por el lado trasero y posteriormente unidos con el otro soporte de acrílico. Ahora, se deben unir los engranajes frontal y trasero al motor dc, junto a su soporte de motor, armando la \textbf{unidad de potencia} mostrada en la Figura \ref{fig:3-Armado_3} .


\begin{figure}[H]
    \centering
    \includegraphics[width=0.9\textwidth]{img/3-Armado_3}
    \caption{Unidad de potencia de la correa transportadora armada, no requiere sujeciones.}
    \label{fig:3-Armado_3}
\end{figure}

\par Luego, procedemos a insertar la unidad de potencia en el rodamiento izquierdo del soporte lateral trasero, mediante el eje que sobresale del engranaje, de acuerdo a la Figura \ref{fig:3-Armado_4} .

\begin{figure}[H]
    \centering
    \includegraphics[width=0.9\textwidth]{img/3-Armado_4}
    \caption{Unidad de potencia incorporada.}
    \label{fig:3-Armado_4}
\end{figure}

\par Luego, para sujetar la unidad de potencia, se debe anclar el soporte inferior a las 2 perforaciones inferiores de 3mm cercanas a esta unidad, tal como se muestra en la Figura \ref{fig:3-Armado_5}. 

\begin{figure}[H]
    \centering
    \includegraphics[width=0.9\textwidth]{img/3-Armado_5}
    \caption{Unidad de potencia empernada con soporte inferior.}
    \label{fig:3-Armado_5}
\end{figure}

\par De manera análoga se repite este paso con el soporte inferior ubicado al otro extremo del soporte acrílico, véase la Figura \ref{3-Armado_6}.

\begin{figure}[H]
    \centering
    \includegraphics[width=1\textwidth]{img/3-Armado_6}
    \caption{Soportes inferiores empernados a soporte de acrílico.}
    \label{fig:3-Armado_6}
\end{figure}

\par La placa de componentes, debe empernarse junto a el driver L298N con pernos M3x8mm y el porta pilas 2x18650 con pernos M3x4mm de cabeza plana, para posibilitar la colocación de las pilas, ver Figura \ref{fig:3-Armado_7}.


\begin{figure}[H]
    \centering
    \includegraphics[width=0.8\textwidth]{img/3-Armado_7}
    \caption{Placa de componentes ensamblada}
    \label{fig:3-Armado_7}
\end{figure}

\par De esta forma, luego se añade la placa de componentes a el ensamble, empernando en ambos soportes inferiores, de acuerdo a lo mostrado en la Figura \ref{fig:3-Armado_8}.


\begin{figure}[H]
    \centering
    \includegraphics[width=1\textwidth]{img/3-Armado_8}
    \caption{Placa de componentes incorporada a la correa transportadora}
    \label{fig:3-Armado_8}
\end{figure}

\par \textbf{IMPORTANTE:} Tras completar este paso, verificar las conexiones iniciales del motor DC, el driver L298N y la fuente de alimentación. Dejar un cable adicional disponible para la futura conexión del interruptor SPST. Conectar de inmediato los tres pines de control del motor DC (ENA, IN1 e IN2). Asimismo, reservar un cable para el retorno de 5V y GND, que servirá para alimentar los servomotores. \\

\par Luego, queda incorporar la \textbf{unidad de soporte} de la correa transportadora, conformada por el eje de 7mm y el cilindro. Para esto, solo hay que colocar el eje de 7mm a presión en el rodamiento opuesto a la unidad de potencia y colocar el cilindro dentro del eje de 7mm, tal como se muestra en la Figura \ref{fig:3-Armado_9} .

\begin{figure}[H]
    \centering
    \includegraphics[width=0.8\textwidth]{img/3-Armado_9}
    \caption{Unidad de soporte incorporada a la correa transportadora}
    \label{fig:3-Armado_9}
\end{figure}

\par Para finalizar con el soporte de acrílico lateral trasero, basta colocar los servomotores en las perforaciones superiores de los soportes de acrílico. Primero hay que fijar los servomotres sin horquillas a los soportes con pernos M3, de acuerdo a la Figura \ref{fig:3-Armado_10}. Debe pasar el cable por el sacado que tiene el soporte antes de acomodarlo.

\begin{figure}[H]
    \centering
    \includegraphics[width=0.6\textwidth]{img/3-Armado_10}
    \caption{Soporte servomotor con servo sg90 empernado.}
    \label{fig:3-Armado_10}
\end{figure}

\par Luego, se debe incorporar los dos servomotores al ensamble actual según se ilustra en la Figura \ref{fig:3-Armado_11}.

\begin{figure}[H]
    \centering
    \includegraphics[width=1\textwidth]{img/3-Armado_11}
    \caption{Servomotores incorporados al ensamble de la correa.}
    \label{fig:3-Armado_11}
\end{figure}

\par Por último, debe añadir los apoyos de las paredes laterales a ambos soportes de acrílico mediante pernos M4x8mm, véase Figura \ref{fig:3-Armado_12}.

\begin{figure}[H]
    \centering
    \includegraphics[width=1\textwidth]{img/3-Armado_12}
    \caption{Apoyos de paredes laterales incorporados al soporte lateral de acrílico.}
    \label{fig:3-Armado_12}
\end{figure}


\par A continuación, se debe colocar la cadena de oruga cuidadosamente entre los dientes de los engranajes pertenecientes a la unidad de potencia y sobre el cilindro de la unidad de soporte, de acuedo a la Figura \ref{fig:3-Armado_13}.

\begin{figure}[H]
    \centering
    \includegraphics[width=1.2\textwidth]{img/3-Armado_13}
    \caption{Cadena de oruga incorporada al ensamble.}
    \label{fig:3-Armado_13}
\end{figure}

\par Luego, se incorpora el otro soporte lateral de acrílico sobre los ejes de la unidad de potencia y la unidad de soporte y se fijan con pernos los soportes inferiores al soporte de acrílico,de acuerdo a la Figura \ref{fig:3-Armado_14}.

\begin{figure}[H]
    \centering
    \includegraphics[width=1\textwidth]{img/3-Armado_14}
    \caption{Soporte lateral de acrílico faltante ensamblado.}
    \label{fig:3-Armado_14}
\end{figure}

\par \textbf{IMPORTANTE:} En este paso, se debe conectar el interruptor SPST de 2 pines utilizando el cable previamente reservado, conectándolo al polo positivo de las pilas para controlar la alimentación de la correa transportadora. A su vez, pasar las conexiones de los servomotores por el orificio opuesto al soporte del switch, asegurándose de que todos los cables queden orientados hacia el frente de la correa para su posterior conexión. El interruptor colocado en la perforación de 20mm se muestra en la Figura \ref{fig:3-Armado_15}.

\begin{figure}[H]
    \centering
    \includegraphics[width=0.8\textwidth]{img/3-Armado_15}
    \caption{Interruptor (switch) SPST de 2 pines incorporado al ensamble.}
    \label{fig:3-Armado_15}
\end{figure}


\par Ahora, colocamos las tapas cubrerodamientos sobre los 4 rodamientos, considerando que estas vienen con los imanes de neodimio 12x4mm pegadas mediante algún pegamento. De esta forma, se obtiene el resultado de la Figura \ref{fig:3-Armado_16}. 

\begin{figure}[H]
    \centering
    \includegraphics[width=0.8\textwidth]{img/3-Armado_16}
    \caption{Tapas cubrerodamientos colocadas}
    \label{fig:3-Armado_16}
\end{figure}

\par Luego, se añaden las palas a los servomotores (recuerde calibrar el cero de los servomotores antes de colocarlas para que el rango de actuación esté bien), ver Figura \ref{fig:3-Armado_17}.


\begin{figure}[H]
    \centering
    \includegraphics[width=0.8\textwidth]{img/3-Armado_17}
    \caption{Palas de servomotores añadidas}
    \label{fig:3-Armado_17}
\end{figure}

